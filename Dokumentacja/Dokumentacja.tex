\documentclass[a4paper,12pt]{article}

\usepackage[utf8]{inputenc}
\usepackage[T1]{fontenc}
\usepackage[polish]{babel}
\usepackage{lmodern}        % lepsze fonty wektorowe
\usepackage[hidelinks,plainpages=false,hypertexnames=false]{hyperref}       % linki w dokumencie
\usepackage{graphicx}       % do wstawiania obrazków
\usepackage{listings}       % do prezentacji kodu źródłowego
\usepackage{amsmath,amssymb} % dodatkowe symbole matematyczne
\usepackage{breakurl}        % do łamania adresów URL
\usepackage{titlesec}        % do ustawiania formatowania tytułów
\usepackage{fancyhdr}        % do numeracji stron
\usepackage{setspace}        % do ustawienia interlinii
\usepackage{enumitem}        % do zaawansowanego formatowania list
\usepackage{pifont}          % do dodania znaczników typu checkmark
\usepackage{array}           % do tabel

% Ustawienia formatowania strony
\usepackage[a4paper,margin=2.5cm]{geometry}

% Numeracja stron u dołu (bez strony tytułowej)
\pagestyle{fancy}
\fancyhf{}
\rfoot{\thepage}
\renewcommand{\headrulewidth}{0pt}

% Czcionka: Times New Roman
\renewcommand{\rmdefault}{ptm}

% Formatowanie tytułów rozdziałów i podrozdziałów
\titleformat{\section}{\bfseries\fontsize{14pt}{14pt}\selectfont}{\thesection}{1em}{}
\titlespacing{\section}{0pt}{12pt}{6pt}

\titleformat{\subsection}{\bfseries\fontsize{13pt}{13pt}\selectfont}{\thesubsection}{1em}{}
\titlespacing{\subsection}{0pt}{6pt}{6pt}

\titleformat{\paragraph}[runin]{\bfseries\fontsize{12pt}{12pt}\selectfont}{\theparagraph}{1em}{}
\titlespacing{\paragraph}{0pt}{6pt}{6pt}

% Interlinia 1.5
\renewcommand{\baselinestretch}{1.5}

% Wcięcie pierwszego wiersza akapitu 0,7 cm
\setlength{\parindent}{0.7cm}

\sloppy % Wymuszanie łamania tekstu

\begin{document}

%-----------------------------------------
% Strona tytułowa (bez numeracji strony)
%-----------------------------------------
\thispagestyle{empty}
\begin{titlepage}
    \centering
    {\Large\textbf{Dokumentacja Techniczna Projektu}}\\[1em]
    {\large \textbf{Projekt: Karaoke Machine}} \\

    \textbf{Autorzy projektu:}

    \vspace{1em}

    \begin{tabular}{@{}l r@{}}
    Dawid Brożyna & 42719 \\
    Jakub Demkowski & 42724\\
    Dominik Gwóżdź & 43189 \\
    Kamil Rudowski & 43256 \\
    \end{tabular}

    \vfill

    {\large \textbf{Wersja dokumentu: 1.0}}\\[0.5em]
    {\large \textbf{Data: \today}}
\end{titlepage}

\tableofcontents
\newpage

%-----------------------------------------
\section{Opis funkcjonalny systemu}
System \textbf{Karaoke Machine} został zaprojektowany w celu organizacji sesji karaoke. Główne funkcjonalności systemu obejmują:

\begin{tabular}{@{}p{0.9\linewidth}p{0.1\linewidth}@{}}
    - Użytkownik może zarejestrować się w systemie & \ding{51} \\
    - Użytkownik może wyszukiwać piosenki (po tytule, autorach, fragmencie tekstu i kategoriach) & \ding{51} \\
    - Użytkownik może uruchomić piosenkę & \ding{51} \\
    \quad - Piosenka może być uruchomiona z zewnętrznego systemu (np. YouTube) & \ding{51} \\
    \quad - Na ekranie muszą pojawiać się słowa piosenki przypisane do konkretnego czasu odtworzenia & \ding{51} \\
    - Użytkownik może zobaczyć tekst piosenki z przypisanymi czasami do konkretnych wierszy & \ding{51} \\
    - Użytkownik może proponować zmianę tekstu lub czasów & \ding{51} \\
    \quad - Zmiany mogą zostać później zaakceptowane lub odrzucone przez administratora & \ding{51} \\
    - Administrator może zarządzać użytkownikami: widzi ich dane, może ich blokować & \ding{51} \\
    - Administrator może zarządzać autorami i kategoriami piosenek & \ding{51} \\
    - Administrator może zarządzać piosenkami & \ding{51} \\
    \quad - Podpinać źródło muzyczne & \ding{51} \\
    \quad - Zarządzać tekstem i znacznikami czasowymi & \ding{51} \\
    \quad - Zarządzać podpowiedziami od użytkowników & \ding{51} \\
    - Rejestracja powinna odbywać się klasycznie poprzez mejla oraz Facebooka & \ding{51} \\
\end{tabular}

%-----------------------------------------
\section{Opis technologiczny}
Projekt wykorzystuje następujące technologie i narzędzia:
\begin{itemize}
    \item \textbf{Język programowania:} PHP 8.3.13
    \item \textbf{Framework:} Laravel 11.30.0
    \item \textbf{Package:} Laravel Socialite 5.17
    \item \textbf{Framework CSS:} Tailwind CSS 3.4.13
    \item \textbf{Serwer:} Apache 2.4
    \item \textbf{Baza danych:} MariaDB/MySQL
    \item \textbf{API:} YouTube Player API Reference for iframe Embeds
    \item \textbf{System operacyjny:} Windows/Linux
\end{itemize}
Architektura systemu opiera się na wzorcu MVC (Model-View-Controller), co zapewnia modularność i łatwość w rozbudowie aplikacji

%-----------------------------------------
\section{Podział obowiązków i odpowiedzialności w zespole}
Podział obowiązków w zespole projektowym wyglądał następująco:
\begin{itemize}
    \item \textbf{Dominik Gwóżdź:} Project manager, Inżynier devops,  Backend
    \item \textbf{Jakub Demkowski:} Frontend
    \item \textbf{Dawid Brożyna:} Tester
    \item \textbf{Kamil Rudowski:} Backend
\end{itemize}

%-----------------------------------------
\section{Instrukcja lokalnego i zdalnego uruchomienia systemu}
\subsection{Lokalne uruchomienie systemu}
Aby uruchomić aplikację lokalnie, wykonaj następujące kroki:
\begin{enumerate}
    \item Klonowanie repozytorium:
    \begin{verbatim}
    git clone https://github.com/dominog125/KaraokeMachine.git
    \end{verbatim}
    \item Instalacja zależności:
    \begin{verbatim}
    composer install
    \end{verbatim}
    \item Konfiguracja pliku .env zgodnie z lokalnymi ustawieniami
    \item Wygenerowanie klucza aplikacji:
    \begin{verbatim}
    php artisan key:generate
    \end{verbatim}
    \item Migracja bazy danych:
    \begin{verbatim}
    php artisan migrate
    \end{verbatim}
    \item Uruchomienie serwera lokalnego:
    \begin{verbatim}
    php artisan serve
    \end{verbatim}
    \item Kompilacja frontendu:
    \begin{verbatim}
    npm run dev
    \end{verbatim}
\end{enumerate}

\subsection{Zdalne uruchomienie systemu}
Aby uruchomić aplikację na serwerze produkcyjnym, wykonaj następujące kroki:
\begin{enumerate}
    \item Skopiuj pliki projektu na serwer
    \item Skonfiguruj środowisko produkcyjne (np. Apache, MariaDB)
    \item Ustaw zmienne środowiskowe w pliku .env
    \item Uruchom migracje bazy danych:
    \begin{verbatim}
    php artisan migrate
    \end{verbatim}
    \item Kompilacja frontendu:
    \begin{verbatim}
    npm run production
    \end{verbatim}
\end{enumerate}

%-----------------------------------------
\section{Wnioski projektowe}
Poniżej przedstawiono wnioski od poszczególnych członków zespołu:
\begin{itemize}
    \item \textbf{Dawid Brożyna:}  Praca nad konfiguracją i testowaniem aplikacji była cennym doświadczeniem
    \item \textbf{Jakub Demkowski:} Projekt umożliwił mi rozwinięcie umiejętności projektowania interfejsów użytkownika
    \item \textbf{Dominik Gwóżdź:} Projekt pozwolił mi udoskonalić umiejętności zarządzania zespołem oraz efektywnego planowania zadań. Dodatkowo nauczyłem się lepiej monitorować ryzyka projektowe.
    \item \textbf{Kamil Rudowski:} Dzięki projektowi nauczyłem się efektywnego wykorzystania frameworka Laravel
\end{itemize}

\end{document}
